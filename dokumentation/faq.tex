\chapter{Appendix A: H�ufig gestellte Fragen} \label{kap_faq}

\textbf{Frage:} Soll der Quellcode von Programmen im Anhang zu finden sein?\\
Antwort: Ja, falls der Code nicht zu umfangreich ist. Ansonsten muss er auf CD bereitgestellt werden, wobei dieses durch entsprechende Hinweise im Text deutlich gemacht werden muss.
\\

\noindent\textbf{Frage:} Ich beziehe mich auf Teile des Quellcodes, aber dieser ist nur auf der CD vorhanden. Ist das g�nstig? \\
Antwort: In diesem Fall sollten diese Teile in sinnvollen Ausschnitten im Text auftreten.
\\

\noindent\textbf{Frage:} Wie soll die Arbeit gebunden werden? \\
Antwort: Bei d�nnen Arbeiten reicht nur eine Klebung. Bei �ber 80 Seiten ist aber eine Heftung plus Klebung notwendig. Es macht keinen guten Eindruck, wenn die Arbeit beim Durchlesen durch die Pr�ferInnen auseinanderf�llt.
\\

\noindent\textbf{Frage:} Sind besondere Erkl�rungen zum geistigen Eigentum notwendig? \\
Antwort: Dieses ist nicht erforderlich, wenn nach den beschriebenen Regeln vorgegangen und zitiert wird. Es ist dann sowieso nicht sinnvoll, seitenlang aus Quellen abzuschreiben.
\\

\noindent\textbf{Frage:} Wie geht man mit firmeninternen Quellen um? \\
Antwort: Diese k�nnen ebenfalls als Literatur angeben werden, z.B.: [7] x,y,z.doc Schriftst�ck Fa. X, Ort Y, internes Aktenzeichen abcde, Abteilung zz
\\

\noindent\textbf{Frage:} Darf die Arbeit farbig gedruckt werden? \\
Antwort: Ja, aber in gleicher Art und Weise bei den beiden Exemplaren.
\\

\noindent\textbf{Frage:} Ich m�chte meinem Hund danken, dass er immer beim Schreiben in meiner N�he war. Ist das m�glich? \\
Antwort: Manche Firmen empfinden Danksagungen und Widmungen vorne in der Abschlussarbeit als unpassend. Wenn eine Danksagung dieser Art erfolgen soll, dann besser am Ende der Arbeit.
\\

\noindent\textbf{Frage:} Ich habe keine Lehrerin als Freundin, die alles durchliest. Was kann ich tun? \\
Antwort: Zun�chst einmal sollte die Rechtschreibpr�fung des Texteditors eingeschaltet werden (und zwar schon zu Beginn des Zusammenschreibens, da die Rechtschreibpr�fung neue Begriffe hinzulernen soll). Zus�tzlich sollte man Geschwister/Eltern/Freunde bitten, die Arbeit durchlesen. Ggf. kann auch die/er BetreuerIn in der Firma kritische Anmerkungen geben, wobei die FirmenmitarbeiterInnen oftmals viel Zeit zum Lesen ben�tigen.
\\

\noindent\textbf{Frage:} Wie soll man mit englischen Begriffen umgehen? \\
Antwort: Es sollte zun�chst versucht werden, diese zu vermeiden. Oftmals gibt es sinnvolle deutsche Begriffe. Falls dieses nicht m�glich ist, sollten die Begriffe im Text erkl�rt werden. Insgesamt muss der Text f�r eine(n) LeserIn mit durchschnittlichen Fachkenntnissen verst�ndlich sein.
\\

\noindent\textbf{Frage:} Wie wird die Arbeit bewertet? \\
Antwort: Jede(r) Pr�ferIn hat daf�r ein eigenes Bewertungsschema, wobei oftmals viele der folgenden Kriterien betrachtet werden: fachlicher Inhalt und Ausdrucksweise, wesentliche und unwesentliche Punkte, Gliederung und Aufbau der Arbeit, Texte und Bilder, Vollst�ndigkeit, Gestaltung Anhang, Literaturverzeichnis mit Querchecks, eigene Kreationen oder "`raubkopiert"', ingenieurm��iges Denken und die Darstellung dieser Aspekte, Betreuungsaufwand. Ggf. zusammen mit der Beurteilung durch eine Firma ergibt sich erfahrungsgem�� ein sehr gutes Bild, wobei die Firmen oft eher schlechter beurteilen. Die Pr�ferInnen bestimmen letztendlich die Note und die Abschlussarbeit ist "`die Urkunde"'.
\\

\noindent\textbf{Frage:} K�nnen die Pr�ferInnen erkennen, wenn Passagen aus dem Internet herauskopiert sind? \\
Antwort: Hierf�r steht eine spezielle Software zur Verf�gung, die alle Passagen der Arbeit mit dem Internet vergleicht. Diese Software wurde f�r Schulen und Hochschulen extra entwickelt.
\\

\noindent\textbf{Frage:} Schreibe ich in der Pr�sensform oder muss ich oft "`wurde gemacht"' und "`dann wurde"' benutzen? \\
Antwort: Hier gibt es unterschiedliche Auffassungen. Einige Studierende bevorzugen die Vergangenheitsform, andere dagegen die Gegenwartsform. Wichtig: Es sollte darauf geachtet werden, die einmal gew�hlte Zeitform konsequent beizubehalten und niemals in der "`Ich-Form"' zu schreiben.
\\

\noindent\textbf{Frage:} Wie sollte das Literaturverzeichnis in der Arbeit organisiert werden? \\
Antwort: Verwenden Sie hierzu eine BibTeX-Datei und tragen Sie die Angaben zu den Literaturstellen ein. Sobald ein Zitat verwendet wird und eine Neukompilation (inklusive der BibTeX-Datei) erfolgt, wird ein entsprechender Eintrag im Literaturverzeichnis erzeugt.
\\

\noindent\textbf{Frage:} Wie kann ich den Index der Referenzen und Abbildungen aktualisieren, wenn ich �nderungen vorgenommen habe? \\
Antwort: Hier hierzu ist eine zweifache Neukompilation erforderlich.
\\



\endinput 