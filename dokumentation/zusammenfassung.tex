\vspace*{2cm}

\begin{center}
    \textbf{Zusammenfassung der Arbeit / Abstract of Thesis}
\end{center}

\vspace*{1cm}

\begin{table}[htb]
  \centering
  \begin{tabular}{lp{10cm}}
    \toprule
    \textbf{Fachbereich:} & \multirow{2}{*}{Elektrotechnik und Informatik} \\
		\textbf{Department:} & \\
    \midrule
		\textbf{Studiengang:} & \multirow{2}{*}{Studiengang} \\
		\textbf{University course:} & \\
		\midrule
		\textbf{Thema:} & \multirow{2}{*}{Anleitung zum Erstellen einer Abschlussarbeit} \\
		\textbf{Subject:} & \\
		\midrule
		\multirow{2}{4cm}{\textbf{Zusammenfassung:}} & Die vorliegende Arbeit ist eine kurze Anleitung zum Schreiben einer Abschlussarbeit, die mit LaTeX erstellt wurde. Sie dient sowohl als Beispiel als auch als Vorlage, da sie die formalen Anforderungen an eine Abschlussarbeit hinsichtlich Layout, Struktur und usw. erf�llt. Die Studierenden k�nnen das Dokument als Referenz verwenden oder auch den ganzen Text in diesem Dokument mit eigenen Inhalten ersetzen sowie Anpassungen entsprechend der eigenen Vorstellungen vornehmen.  \\
		\midrule
		\multirow{2}{4cm}{\textbf{Abstract:}} & The following work is a short guide for writing a thesis. It was created with LaTeX. This document serves both as example and as a thesis template following the formal requirements of the final thesis, such as the layout, font-style, structure, etc. Students can use this template as a reference guide or replace all text in the front part of this template with their own information and customize it to suit their needs.  \\
		\midrule
		\textbf{Verfasser:} & \multirow{2}{*}{Vorname Nachname} \\
		\textbf{Author:} & \\
		\midrule
		\textbf{Betreuender ProfessorIn:} & \multirow{2}{*}{Prof. Dr. Vorname Nachname} \\
		\textbf{Attending professor:} & \\
		\midrule
		\textbf{WS / SS:} & SS 20XX \\
		\bottomrule
  \end{tabular}
\end{table}

