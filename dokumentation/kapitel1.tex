\chapter{Einleitung} \label{kap_einleitung}

\section{Hintergrund}

F�r viele Studierende ist die Erstellung einer Abschlussarbeit eine neue Herausforderung, so dass sie sich erst mit allen Randbedingungen einer solchen Arbeit vertraut machen m�ssen. Auch wenn es eine ganze Reihe von allgemeiner Ratgeberliteratur (wie z.B.~ \cite{les06}) gibt, hat doch jede Hochschule wieder etwas andere formale Abl�ufe und Anforderungen an die Arbeit selbst (z.B.~vorgeschriebene Seiten am Anfang), die ber�cksichtigt werden m�ssen. In dieser Ausarbeitung werden daher die speziellen Randbedingungen der FH L�beck ber�cksichtigt.

WICHTIG: Das vorliegende Dokument ist keine rechtliche Grundlage. Die Pr�fungsordnung und allgemeine Rechtslage ist zu beachten. Der Ablauf der Bachelorarbeit wird durch die Pr�fungsordnung geregelt. Die Vorgaben und Hinweise des/r Erstpr�ferIn sind zudem ma�geblich.

\section{Ziele der Arbeit}

Dieses Dokument dient dazu, Studierende aus dem Fachbereich E+I bei der Erstellung von Abschlussarbeiten zu unterst�tzen. Dabei hat es eine Doppelfunktion: Einmal als Sammlung von Richtlinien und Empfehlungen, die f�r die Erstellung der Abschlussarbeit relevant sind, zum anderen auch als Vorlage geschrieben mit LaTeX (unter Verwendung von MikTeX und dem Editor TeXstudio in Microsoft Windows).

Die Vorlage wurde so geschrieben, dass sie die Formatierungsanforderungen und -em\-pfeh\-lun\-gen f�r das Schreiben einer Abschlussarbeit erf�llt. Die Studierenden haben die M�glichkeit, dieses Dokument als Referenz zu verwenden oder den ganzen Text in diesem Dokument mit den eigenen Inhalten auszutauschen und weitere Anpassungen nach Bedarf vorzunehmen.

\section{Gliederung des weiteren Dokumentes}

Die weiteren Kapitel dieses Dokumentes sind wie folgt gegliedert. In Kapitel \ref{kap_erstellung} wird der Ablauf der Erstellung der Arbeit erkl�rt, wobei auch das Kolloquium mit einbezogen wird. Die Strukturierung der Arbeit wird unter Einbeziehung eines typischen Beispiels in Kapitel \ref{kap_strukturierung} besprochen. In Kapitel \ref{kap_schreiben} werden Anforderungen an das Erstellen der Arbeit, wie z.B. Schreibstil oder Zitierungen, erl�utert. Eine Zusammenfassung der Arbeit und eine kurzer Ausblick auf zuk�nftige Verbesserungsm�glichkeiten sind im Kapitel \ref{kap_zusammenfassung} zu finden.

\endinput
